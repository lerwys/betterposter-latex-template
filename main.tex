%%%% Better Poster latex template example v1.0 (2019/04/04)
%%%% GNU General Public License v3.0
%%%%
%%%% Tom Kocmi - UFAL
%%%% https://github.com/tomkocmi/betterposter-latex-template
%%%%
%%%% Rafael Bailo
%%%% https://github.com/rafaelbailo/betterposter-latex-template
%%%%
%%%% Original design from Mike Morrison
%%%% https://twitter.com/mikemorrison

%%%% REMOVE lanscape FOR VERTICAL POSTER
%\documentclass[a0paper,fleqn,landscape]{betterposter}
\documentclass[a0paper,fleqn]{betterposter}

%% Setting the width of columns
% Left column
%\setlength{\leftbarwidth}{0.25\paperwidth}
% Right column
%\setlength{\rightbarwidth}{0.25\paperwidth}

%% Setting the column margins
% Horizontal margin
%\setlength{\columnmarginvertical}{0.05\paperheight}
% Vertical margin
%\setlength{\columnmarginhorizontal}{0.05\paperheight}
% Horizontal margin for the main column
%\setlength{\maincolumnmarginvertical}{0.15\paperheight}
% Vertical margin for the main column
%\setlength{\maincolumnmarginhorizontal}{0.15\paperheight}

%% Changing font sizes
% Text font
%\renewcommand{\fontsizestandard}{\fontsize{28}{35} \selectfont}
% Main column font
%\renewcommand{\fontsizemain}{\fontsize{100.02}{189.70} \selectfont}
%\renewcommand{\fontsizemain}{\fontsize{28}{35} \selectfont}
% Title font
%\renewcommand{\fontsizetitle}{\fontsize{28}{35} \selectfont}
% Author font
%\renewcommand{\fontsizeauthor}{\fontsize{28}{35} \selectfont}
% Section font
%\renewcommand{\fontsizesection}{\fontsize{28}{35} \selectfont}

% Use this for caption-line in figures
\newcommand{\fontsizecaption}{\fontsize{32.00}{45.33} \selectfont}

%% Changing font sizes for a specific text segment
% Place the text inside brackets:
% {\fontsize{28}{35} \selectfont Your text goes here}

%% Changing colours
% Background of side columns
%\renewcommand{\columnbackgroundcolor}{black}
% Font of side columns
%\renewcommand{\columnfontcolor}{gray}
% Font color of acknowledgement
\renewcommand{\acknowledgementbackgroundcolor}{white}
% Background of main column
%\renewcommand{\maincolumnbackgroundcolor}{white}
%\renewcommand{\maincolumnbackgroundcolor}{theory}
\renewcommand{\maincolumnbackgroundcolor}{imperialblue}
%\renewcommand{\maincolumnbackgroundcolor}{methods}
%\renewcommand{\maincolumnbackgroundcolor}{intervention}
% Font of main column
\renewcommand{\maincolumnfontcolor}{white}

\gdef\title{The Title of Your Amaizing Paper}
\gdef\author{Mike Morrison, Rafael Bailo, Tom Kocmi and Lucas Russo}
\gdef\email{dig@lnls.br}
\gdef\urlQRcode{https://github.com/lnls-dig}
\gdef\conferenceLogo{img/icalepcs_2019_logo}
\gdef\acknowledgement{Acknowledgements of your paper. This template is based on Mike Morrison's idea of Better Poster. It was modified to latex by Rafael Bailo and modified for UFAL purposes by Tom Kocmi.}

\begin{document}
\betterposter{

%%%%%%%% MAIN COLUMN START

\textbf{Main finding} goes here,
\\translated into \textbf{plain English}.
\\\textbf{Emphasize} the important!

%%%%%%%% MAIN COLUMN END
}{
%%%%%%%% LEFT COLUMN START - SECTION FOR PEOPLE TO READ BY THEMSELVES

\section{Introduction}
This section is for participants to read through by themselves without your
interaction. Remember, that you are standing on the right side of the poster,
thus on the left side can several people stand and read this section without
your interuption.

\vspace*{2.0em}

First specify what was your problem, maybe add some graph or illustration
\vspace*{2.0em}
\begin{center}
% Linear regression
% Author: Henri Menke
% Retrieved from: http://www.texample.net/tikz/examples/linear-regression/
\includegraphics[width=\linewidth]{img/tikzexample1}
\end{center}

\section{Methods}

Describe your contribution, what is your main goal. Add explanatory equation, etc.

\begin{equation}
\int_a^b f(x)\,\mathrm{d}x = F(b)-F(a).
\end{equation}

\section{Results}

Be brief about your results, maybe list them as individual items.

\begin{itemize}
\item The first item.
\item The second item.
\item The third item.
\end{itemize}

\vspace*{3.0em}

Note: if you want to scale a graph or something to the full width of a column, use:

\vspace*{1.0em}

\resizebox{\columnwidth}{!}{YOUR OBJECT}

To use a figure spanning multiple columns use:

\multicolinterrupt{
\vspace*{2.0em}
\centering
\includegraphics*[width=\textwidth]{img/deployment-pipeline.pdf}
\vspace*{-1.0em}
%{\fontsizecaption Example Deployment Workflow.}
}

%%%%%%%% LEFT COLUMN END
}{
%%%%%%%% RIGHT COLUMN START
% you can space sections by \vspace*{3.0em}

Here you can add \textbf{supplementary material}. Remember, this section will be
right next to you.
Use this section for supplementary material that is needed for YOUR oral
explanation. Do not care about titles or naming the sections because you will
tell the participants what is what.

\vspace*{1.0em}

For instance, a table:

\resizebox{\columnwidth}{!}{
\begin{tabular}{lll}
 Country Name & ISO ALPHA 2  & ISO ALPHA 3 \\
 \hline
 Afghanistan   & AF & AFG \\
 Aland Islands & AX & ALA \\
 Albania       & AL & ALB \\
 Algeria       & DZ & DZA \\
 American Samoa& AS & ASM \\
 Andorra       & AD & AND \\
 Angola        & AO & AGO \\
 \hline
\end{tabular}
}

\vspace*{1.0em}

Or very important JaCoW logo:
\begin{center}
\includegraphics[width=\textwidth]{img/jacow-logo}
% Retrieved from https://www.jacow.org/About/PressKit
\end{center}

%%%%%%%% RIGHT COLUMN END
}
\end{document}
